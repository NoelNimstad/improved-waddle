\documentclass[a4paper]{article}
\usepackage[top=2cm, left=2cm, bottom=2cm, right=2cm]{geometry}
\usepackage{physics}
\usepackage{amsthm}
\usepackage{amsmath}
\usepackage{amssymb}
\usepackage{setspace}
\usepackage[dvipsnames]{xcolor}
\usepackage{pagecolor}
\usepackage[some]{background}
\usepackage{tocloft}
\usepackage{esvect}
\usepackage[most]{tcolorbox}
\newtcolorbox{note}{boxrule=0pt, colback=white, frame hidden, sharp corners, enhanced, borderline west={1pt}{0pt}{RoyalBlue}, breakable=true}
\renewcommand{\cftsecfont}{\color{white}}
\renewcommand{\cftsubsecfont}{\color{white}}
\doublespacing
\renewcommand{\arraystretch}{0.604}
\newcommand{\point}[2]{\langle#1\,,#2\rangle}
\renewcommand{\point}[3]{\left\langle#1\,,#2\,,#3\right\rangle}
\renewcommand{\familydefault}{qhv}
\renewcommand{\vector}[2]{\begin{pmatrix}#1\\#2\end{pmatrix}}
\renewcommand{\vector}[3]{\begin{pmatrix}#1\\#2\\#3\end{pmatrix}}
\renewcommand{\vec}[1]{\mathbf{#1}}
\newenvironment{example}{\begin{note}\large\textbf{Example}\normalsize\newline}{\end{note}}

\begin{document}
\begin{titlepage}
    \pagecolor{RoyalBlue}
    \color{white}
    \vspace*{5em}
    \Huge\textbf{Improved Waddle}

    \Large\textbf{For the IB Math AA HL course}

    \vspace*{5em}
    \LARGE By \textbf{Noël M. Nimstad}
    \vspace*{17.5em}
    \normalsize
    \tableofcontents
\end{titlepage}
\pagecolor{white}
\newpage
\section{Vectors}
\subsection{Planes}
Fundementally, a point $\vec{P}$ lies on a plane $\Pi$ if it satisfies the equation:
$$
    \vv{AP}=\lambda\vv{AB}+\mu\vv{AC}\qc\lambda,\mu\in\mathbb{R}
$$
where $\vec{A},\vec{B}$ and $\vec{C}$ are three other points on the plane.
The normal vector $\vec{n}$ of a plane is found by taking the cross product of two vectors on the plane:
$$
    \vec{n}=\vv{AB}\cross\vv{AC}
$$
There are three main formulaeic ways of describe a plane, these are:
\begin{itemize}
    \item The vector equation:
    $$
    \vec{r}=\vec{a}+\lambda\vec{u}+\mu\vec{v}\qc\lambda,\mu\in\mathbb{R}
    $$
    where $\vec{a},\vec{b}$ and $\vec{c}$ are three position vectors in space.
    This form is used to generate positions $\vec{r}$ through the setting of $\lambda$ and $\mu$.
    This form is equivalently stated as:
    $$
        \vector{x}{y}{z}=\vector{a_1}{a_2}{a_3}+\lambda\vector{u_1}{u_2}{u_3}+\mu\vector{v_1}{v_2}{v_3}\qc\lambda,\mu\in\mathbb{R}
    $$
    \item The parametric form is obtained by making the vector equation in to a system of equations.
    \item The Cartesian form:
    $$
        ax+by+cz=d\qc a,b,c,d\in\mathbb{R}
    $$
    This form is obtained by removing $\lambda$ and $\mu$ from the parametric form's system of equations.
    This form is homogenous to the equation of a straight line ($ax+by=c$) in 2 dimensional Cartesian $x$-$y$ space, extended for 3 variables.
    If we know the normal vector $\vec{n}$ of the plane and one point $\vec{a}$, this form is also equivalently stated as (with $\vec{p}$ being any arbritrary point):
    $$
        \vec{n}\cdot\vec{p}=\vec{n}\cdot\vec{a}
    $$
    The normal vector of a plane stated in Cartesian form $ax+by+cz=d$ is:
    $$
        \vec{n}=\vector{a}{b}{c}
    $$
\end{itemize}
\begin{example}
    The point $A\point{0}{2}{1}$, $B\point{3}{0}{-1}$ and $C\point{-2}{1}{1}$ lie on the plane $\Pi$.
    The vector equation of the plane is given by first finding $\vec{u}=\vv{AB}$ and $\vec{b}=\vv{AC}$:
    \begin{align*}
        \vec{u}=\vv{AB}&=\point{3}{0}{-1}-\point{0}{2}{1}=\vector{3}{-2}{-2}\\
        \vec{v}=\vv{AC}&=\point{-2}{1}{1}-\point{0}{2}{1}=\vector{-2}{-1}{0}
    \end{align*}
    Thus the vector equation is given as:
    $$
        \vector{x}{y}{z}=\vector{0}{2}{1}+\lambda\vector{3}{-2}{-2}+\mu\vector{-2}{-1}{0}\qc\lambda,\mu\in\mathbb{R}
    $$
    Analagously, the parametric form is given by the system of equations:
    $$
        \left\{\begin{array}{ll}
            x=3\lambda-2\mu\\
            y=2-2\lambda-1\mu\\
            z=1-2\lambda\\
        \end{array}\right.
    $$
    To convert to Cartesian form, we need to eliminate $\mu$ and $\lambda$ from the system.
    We see:
    $$
        x-2y=3\lambda-2\mu-4+4\lambda+2\mu=-4+7\lambda
    $$
    Thus:
    \begin{align*}
        x-2y+3.5z&=-4+7\lambda+3.5-7\lambda=-0.5\\
        \implies2x-4y+7z&=-1
    \end{align*}
    The normal vector is therefore also given as:
    $$
        \vec{n}=\vector{2}{-4}{7}
    $$
\end{example}
If the normal vector of two planes $\vec{n}_1$ and $\vec{n}_2$ are collinear ($\vec{n}_1=a\vec{n}_2,\,a\in\mathbb{R}$), then the planes are parallel.
Otherwise, the planes intersect at a line given by two different methods.
The first is the solution to the system of equations of the planes' Cartesian forms letting one of the coordinates equal some variable $\lambda$.
The other is taking the cross product of the two normal vector $\vec{n}_1\cross\vec{n}_2$, which gives the direction vector of the line.
\begin{example}
    Let $\Pi_1:2x-4y+7z=1$ and $\Pi_2:-x+y+2z=0$.
    The normal vectors of the planes are:
    \begin{align*}
        \vec{n}_1&=\vector{2}{-4}{7}\\
        \vec{n}_2&=\vector{-1}{1}{2}
    \end{align*}
    $\because\nexists a\in\mathbb{R}$ s.t. $\vec{n}_1=a\vec{n}_2$, $\neg(\Pi_1\parallel\Pi_2)$.

    \textbf{Method 1}
    The planes' intersection is given at the line which is the solution to the system, letting $z=\lambda$:
    \begin{align*}
        &\left(\begin{array}{ll}
            2x-4y=1-7\lambda\\
            -x+y=-2\lambda
        \end{array}\right)\\
        &\sim\left(\begin{array}{ll}
            2x-4y=1-7\lambda\\
            -2x+2y=-4\lambda
        \end{array}\right)_{2R_2\rightarrow R_2}\\
        &\sim\left(\begin{array}{ll}
            -2y=1-11\lambda\\
            -2x=1-15\lambda
        \end{array}\right)_{R_1+R_2\rightarrow R_1,\,2R_2+R_1\rightarrow R_2}\\
        &\sim\left(\begin{array}{ll}
            y=\frac{11\lambda-1}{2}\\
            x=\frac{15\lambda-1}{2}
        \end{array}\right)_{-\frac{R_1}{2}\rightarrow R_1,\,-\frac{R_2}{2}\rightarrow R_2}
    \end{align*}
    The intersection is thus the line:
    $$
        L:\lambda\mapsto\point{\frac{15\lambda-1}{2}}{\frac{11\lambda-1}{2}}{\lambda}
    $$
    
    \textbf{Method 2}
    The direction vector $\vec{d}$ is given by:
    \begin{align*}
        \vec{d}=\vec{n}_1\cross\vec{n}_2&=\vector{2}{-4}{7}\cross\vector{-1}{1}{2}\\
        &=\left\lvert\begin{matrix}
            \hat{\imath} & \hat{\jmath} & \hat{k} \\
            2 & -4 & 7 \\
            -1 & 1 & 2
        \end{matrix}\right\rvert=\hat{\imath}\left\lvert\begin{matrix}
            -4 & 7\\
            1 & 2
        \end{matrix}\right\rvert-\hat{\jmath}\left\lvert\begin{matrix}
            2 & 7\\
            -1 & 2
        \end{matrix}\right\rvert+\hat{k}\left\lvert\begin{matrix}
            2 & -4\\
            -1 & 1
        \end{matrix}\right\rvert\\
        &=-15\hat{\imath}-11\hat{\jmath}-2\hat{k}
    \end{align*}
    We then find a point common on both planes by choosing $z=0$ and their equating equations (using $\Pi_1-1$):
    \begin{align*}
        2x-4y-1&=-x+y\\
        3x-5y&=1
    \end{align*}
    We can then choose the point $\point{\frac{1}{3}}{0}{0}$.
    The equation can then finally be written as:
    $$
        L:\lambda\mapsto\point{\frac{1}{3}}{0}{0}+\point{15}{11}{2}\lambda
    $$
\end{example}

\section{Probability}
\subsection{Normally distributed variables}
If a variable $X$ is normally distributed with a mean of $\mu$ and standard deviation of $\sigma^2$, then it is denoted:
$$
    X\sim N(\mu,\sigma^2)
$$
It is often useful (as will be shown) to shift the distribution of $X$, such that it has standardised means and deviations.
This is what the distribution $Z$ is, where:
$$
    Z=\frac{X-\mu}{\sigma}
$$
Since $Z$ has a known mean of $0$ and a deviation of $1$ (see above equation), we can utilise the inverse normal distribution to solve probability problems in which the mean and deviation need to be found out.

\end{document}